%\documentclass[12pt,reqno]{amsart}
%\usepackage[T2A]{fontenc}				% Поддержка русских букв
%\usepackage[utf8]{inputenc}				% Кодировка utf8
%\usepackage[english, russian]{babel}	% Языки: русский,
%\smartqed
%\usepackage{biblatex}
%\usepackage{apacite}
%\usepackage[numbers]{natbib}
%\usepackage{amsfonts}
%\usepackage{amsmath}
%\usepackage{amsthm}
%\usepackage{mathrsfs}
%\usepackage{amssymb}
%\usepackage{mathbbol}
%\usepackage{stmaryrd}
%\usepackage{graphicx}
%\usepackage{epstopdf}
%\usepackage{hyperref}
%\usepackage{textcomp}
%\usepackage[margin=1in]{geometry}

%\setlength{\topmargin}{-0.1cm}
%\setlength{\oddsidemargin}{1.4cm}
%\setlength{\evensidemargin}{1.4cm}
%\setlength{\textwidth}{15.5cm}
%\setlength{\textheight}{22cm}%\unitlength=1cm
%\renewcommand{\theequation}{\thesection.\arabic{equation}}
%\newtheorem{theorem}{Theorem}
%\newtheorem{lemma}{Lemma}
%\newcommand{\dom}{{\rm dom\,}}
%\newcommand{\co}{{\rm co\,}}
%\theoremstyle{definition}
%\newtheorem{definition}{Definition}
%\newtheorem{assumption}{Assumption}
%\newtheorem{remark}{Remark}
%\newtheorem{cor}{Corollary}
%\newtheorem{conjecture}{Conjecture}
\chapter{Оптимальные стратегии производства и назначения цен в динамической модели фирмы-монополиста} \label{chapt3}

\section{Введение} \label{sec:3.1}
\setcounter{equation}{0}
В последние десятилетия в экономической литературе наметился значительный интерес к моделям фирм, которые координируют стратегии производства и назначения цены на продукцию: см. обзоры \cite{EliSte93,YanGil04,CheLev12}. В этих моделях возможность влиять на спрос с помощью выбора цены может значительно изменять оптимальные уровни запасов. В данной главе изучаются оптимальные стратегии  фирмы, максимизирующей прибыль в рамках детерминированной модели с непрерывным временем.

Предположим, что фирма может производить некоторый товар с интенсивностью $\alpha\ge 0$. Пусть $C(\alpha)$ --- соответствующие затраты на производство. Будучи монополистом, фирма может установить любую цену $p\ge 0$ за единицу товара. Спрос $q=D(p)$ считается известной строго убывающей функцией цены. Элементарная, но естественная задача состоит в максимизации мгновенной прибыли:
\begin{equation} \label{eq:3.1.1}
 R(q)-C(q)\to\max_{q\ge 0},\quad R(q)=q D^{-1}(q).
\end{equation}
Эта простая модель монополии хорошо известна: см., например, \cite[Глава 14]{Var92}. Для положительного оптимального решения $\widehat q$ уравнения (\ref{eq:3.1.1}) предельный доход $R'(\widehat q)$ совпадает с предельными затратами $C'(\widehat q)$.

В данной главе нас интересует обобщение этой модели на случай непрерывного времени. Мы предполагаем что фирма может непрерывно производить и продавать некоторый товар. Цель фирмы состоит в максимизации дисконтированного дохода в случае бесконечного горизонта. Предполагается, что отложенный спрос недопустим, т.е. уровень запаса товара должен быть неотрицательным. Предварительная формулировка соответствующей задачи оптимального управления состоит в следующем:
\begin{equation} \label{eq:3.1.2}
\int_0^\infty e^{-\beta t} (R(q_t)-C(\alpha_t))\,dt\to\max,
\end{equation}
\begin{equation} \label{eq:3.1.3}
 \dot X_t=\alpha_t-q_t,\quad X_0=x\ge 0;\quad X_t\ge 0,\quad t\ge 0.
\end{equation}
Назовем эту задачу \emph{выпуклой}, если функция $R$ выпукла, а $C$ --- вогнута.

В отличие от существующей литературы, мы не вводим затраты на хранение. Причина связана со структурой оптимальных стратегий: даже без учета таких затрат оптимальный запас продукции монотонно убывает, достигает нуля за конечное время и навсегда остается на этом уровне. Фактически, фирме не нужен склад. Следует правда отметить, что в случае невыпуклой функции затрат $C$ может потребоваться использовать распределенные (рандомизированные) стратегии производства, чтобы  удерживать товарный запас $X$ на нулевом уровне и удовлетворить оптимальный спрос. На практике вместо распределенных управлений можно использовать приближенно оптимальные стратегии, соответствующие производственным циклам, при которых запас колеблется вблизи $0$. Таким образом, существует потребность в <<небольшом>> складе.

Первая модель производства/назначения цены с непрерывным временем того же типа, что и (\ref{eq:3.1.2}), (\ref{eq:3.1.3}), была предложена в \cite{Pek74}. Горизонт в \cite{Pek74} конечен, а кривая спроса линейна (и зависит от времени). В 1980-х данное направление исследований было продолжено в \cite{ThoSetTen84,FeiHar85,EliSte87}. Работы \cite{ThoSetTen84,EliSte87} также были направлены на изучение конечного горизонта, при этом дисконтирование даже не вводилось. В \cite{FeiHar85} был рассмотрен случай бесконечного горизонта, но вместо фазового ограничения $X_t\ge 0$ авторы рассматривали штрафную функцию. До работы \cite{FeiHar85} модель с бесконечным горизонтом рассматривалась в \cite{ArvMos81,ArvMos82}. Хотя работы \cite{ArvMos81,ArvMos82} не являются математически строгими, они содержат интересные экономические идеи. В частности, авторы подчеркивают, что если функция затрат невыпукла, то возможность хранить продукцию может привести к превосходству циклической производственной стратегии над статической. Пример из \cite{ArvMos82} анализируется ниже.

Наиболее близкой к рассматриваемой постановке задачи является работа \cite{ChaJouTah03}. Хотя модель \cite{ChaJouTah03} предполагает строго положительные затраты на хранение и относится к случаю конечного горизонта, ряд ключевых выводов остаются верными и для нашей модели. В предположении выпуклости задачи, было установлено, что оптимальная стратегия обычно распадается на три стадии: (i) продается избыточный запас, (ii) производство запущено, но продажи преобладают и запас по-прежнему уменьшается, (iii) запас ликвидирован, интенсивности производства и продаж равны и максимизируют мгновенную прибыль (\ref{eq:3.1.1}). Было подчеркнуто, что накопление запаса не является оптимальным, а оптимальные траектории производства и цены являются неубывающими.

В заключительных замечаниях работы \cite{ChaJouTah03} предлагается рассмотреть невыпуклые функции затрат с целью объяснения явления накопления товарного запаса. Фактически, это в точности является предметом работы, представленной в данной главе. Наши выводы носят смешанный характер: в сочетании с дисконтированием \emph{невыпуклости могут объяснить производственные циклы, но не накопление товарных запасов}.

Другой мотивацией настоящего исследования являются эмпирические наблюдения, указывающие на то, что нередко фирмы  фирмы <<действуют в области убывающих предельных затрат>> (см. \cite{Ram91}). Как отмечено в \cite{Ram91}, невыпуклости функции производственных затрат могут объяснить <<волатильность производства относительно цен>>. Хорошо известная идея проста. Пусть, например, существуют три возможных уровня интенсивности производства: $\alpha\in\{0,1,2\}$ с затратами $C(0)=0$, $C(1)=1$, $C(2)=3/2$. Легко видеть, что для того, чтобы произвести две единицы товара за две единицы времени, лучше производить с интенсивностью $\alpha=2$ на первом временн\'{о}м интервале и ничего не производить на втором, чем производить одну единицу на каждом временном интервале. Если интенсивность спроса равна одной единице, то фирме требуется склад. В рассматриваемой динамической модели приближенная оптимальная стратегия аналогичного характера будет приведена ниже.

Работа построена следующим образом. В разделе \ref{sec:3.2} выводится представление функции Беллмана и исследуется ее регулярность (теорема \ref{th:3.1}). Доказано, что функция Беллмана $v$ непрерывно дифференцируема и строго вогнута, даже если задача не является выпуклой. Отметим, что переход к вогнутой (соотв., выпуклой) оболочке функции $R$ (соотв., $C$) не меняет гамильтониана и, следовательно, функцию Беллмана. Та же функция Беллмана соответствует задаче с рандомизированными управлениями (теорема \ref{th:3.2}).

Случай нулевого начального запаса рассматривается в разделе \ref{sec:3.3}. Для выпуклой задачи оптимальность статической стратегии  $\alpha_t=q_t=\widehat u$, где $\widehat u$ максимизирует мгновенную прибыль (\ref{eq:3.1.1}), была доказана в \cite{ChaJouTah03}. В теореме \ref{th:3.3} приводится необходимое и достаточное условие оптимальности этой стратегии в общем случае. Чтобы дать экономическую интерпретацию этого условия разделим фирму на отделы производства и продаж. Статическая стратегия $\widehat u$ является оптимальной тогда и только тогда, когда существует <<теневая цена>> $\eta$ такая, что
интенсивность $\widehat u$ оптимальна для обоих отделов, продающих товар по этой цене. Наименьшая теневая цена совпадает с наименьшей точкой минимума гамильтониана и с предельной непрямой полезностью $v'(0)$ нулевого запаса. Конечно, данное условие выполняется для выпуклой задачи (теорема \ref{th:3.4}). Если это условие нарушается, мы строим оптимальную распределенную стратегия (теорема \ref{th:3.5}) и приближенно оптимальную обычную стратегию, которая индуцирует циклическое изменение уровня торгового запаса.

В разделе \ref{sec:3.4} мы даем полное описание оптимальных стратегий для положительного начального запаса $x>0$ (теорема \ref{th:3.6}). Здесь имеются две основных стадии. На первой стадии запас строго убывает и достигает нуля. Длительность  $\tau$ этой стадии конечна и вычислена явно. Она зависит только от отношения предельных значений $v'(0)$, $v'(x)$ непрямой полезности $v$ и коэффициента дисконтирования $\beta$. На второй стадии запас удерживается на нулевом уровне с помощью распределенных или обычных стратегий производства и продажи, рассмотренных в разделе \ref{sec:3.3}. Следует отметить, что использование распределенных стратегий может потребоваться только на второй стадии.

В разделе \ref{sec:3.5} мы рассматриваем простой пример выпуклой задачи с линейной функцией затрат $C(\alpha)=c\alpha$, $\alpha\in [0,\overline\alpha]$. Процесс производства должен начаться до того как запас продукции закончится, если и только если цена $c$ за единицу продукции меньше чем наименьшая теневая цена $\zeta$.

В разделе \ref{sec:3.6} мы рассматриваем пример \cite{ArvMos81}, где функция спроса $D$ является линейной, а функция затрат $C$ вогнута для малых значений $\alpha$ и выпукла для больших. Согласно \cite{ArvMos81,ArvMos82} имеются три основных случая. (i) Фирма не существует, т.е. оптимально просто продать весь начальный запас товара. Стадия производства не начинается. (ii) За стадией чистой продажи следуют производственные циклы (т.е. распределенная стратегия производства). (iii) Производство запускается до того как запас товара будет исчерпан, и обычная статическая стратегия является оптимальной, когда запас закончится. По сравнению с \cite{ArvMos81}, мы указывем точные значения параметров, соответствующие этим случаям.

\section{Представление функции Беллмана} \label{sec:3.2}
%\setcounter{equation}{0}
Предположим, что фирма может производить некоторый товар с интенсивностью $\alpha_t\in A$, где $A$ --- замкнутое подмножество $\mathbb R_+=[0,\infty)$. Будучи монополистом, фирма может устанавливать цену $p_t\ge 0$ за единицу товара.
Предполагая, что интенсивность спроса является известной строго убывающей функцией цены: $q=D(p)$, удобно считать, что фирма динамически выбирает интенсивность спроса  $q_t\in Q$. Множество $Q\subset\mathbb R_+$ предполагается компактным. Уровень товарного запаса $X$ удовлетворяет уравнению
\begin{equation} \label{eq:3.2.1}
 X_t=x+\int_0^t (\alpha_s-q_s)\,ds,\quad t\ge 0.
\end{equation}
Пусть отложенный спрос недопустим: $X_t\ge 0$, и множества $Q$, $A$ удовлетворяют следующим условиям:
\begin{equation} \label{eq:3.2.2}
 0\in A\cap Q,\quad A\backslash\{0\}\neq\emptyset,\quad Q\backslash\{0\}\neq\emptyset.
\end{equation}
Решение уравнения (\ref{eq:3.2.1}) также будем обозначать через $X^{x,\alpha,q}$.

Пусть $R(q)=qp=qD^{-1}(q)$ --- функция мгновенного дохода, и $C(\alpha)$ --- функция мгновенных затрат. Цель фирмы состоит в том, чтобы максимизировать дисконтированную прибыль на бесконечном горизонте:
$$ \int_0^\infty e^{-\beta t} (R(q_t)-C(\alpha_t))\,dt,\quad \beta>0.$$

Предполагается, что $R:Q\mapsto\mathbb R_+$ непрерывна, $R(0)=0$, и $C:A\mapsto\mathbb R_+$ --- неубывающая непрерывная функция. Если $A$ является неограниченным, то мы дополнительно предполагаем, что $C$ --- $1$-коэрцитивная функция:
\begin{equation} \label{eq:3.2.3}
 C(\alpha)/\alpha\to +\infty,\quad A\ni\alpha\to+\infty.
\end{equation}

Обозначим через $\mathscr A(x)$ множество всех борелевских функций $\alpha:\mathbb R_+\to A$, $q:\mathbb R_+\to Q$ таких, что уровень товарного запаса (\ref{eq:3.2.1}) неотрицателен. Функция Беллмана $v$ определяется следующим образом
\begin{equation} \label{eq:3.2.4}
v(x)=\sup_{(\alpha,q)\in\mathscr A(x)}\int_0^\infty e^{-\beta t} (R(q_t)-C(\alpha_t))\,dt, \quad x\ge 0.
\end{equation}

Введем гамильтониан
\begin{equation} \label{eq:3.2.5}
H(z)=\widehat R(z)+\widehat C(z),\quad \widehat R(z)=\sup_{q\in Q}\{R(q)-qz\},\quad
\widehat C(z)=\sup_{\alpha\in A}\{\alpha z-C(\alpha)\}.
\end{equation}
Заметим, что функции $\widehat R$, $\widehat C$ выпуклы и конечны на $\mathbb R$. Следовательно, они непрерывны. Напомним (см. \cite{Son86}), что ограниченная равномерно непрерывная функция $u:\mathbb R_+\mapsto\mathbb R$ называется \emph{вязкостным решением с ограничениями (constrained viscosity solution: CVS)}  уравнения Гамильтона-Якоби-Беллмана (HJB)
\begin{equation} \label{eq:3.2.6}
 \beta u(x)-H(u'(x))=0,\quad x\ge 0,
\end{equation}
если для любого $x>0$ (соотв., $x\ge 0$) и для любой тестовой функции $\varphi\in C^1(\mathbb R_+)$ таких, что $x$ --- точка минимума (соотв., максимума) функции $u-\varphi$ на $(0,\infty)$ (соотв., на $[0,\infty)$), верно неравенство:
$$  \beta u(x)-H(\varphi'(x))\ge 0 \quad (\text{соотв.,}\ \le 0)$$
Заметим, что в работе \cite{Son86} рассматривается задача минимизации, и данное выше определение в соответствующим образом модифицировано. Используя терминологию вязкостных решений, можно перефразировать это определение, сказав что $u$ является вязкостным суперрешением (\ref{eq:3.2.6}) на $(0,\infty)$ и вязкостным субрешением на $[0,\infty)$.

Легко видеть, что функция $u\in C^1(\mathbb R_+)$ является CVS уравнения (\ref{eq:3.2.6}), если и только если
\begin{align}
\beta u(x)&=H(u'(x)),\quad x>0, \label{eq:3.2.7}\\
 \beta u(0)&\le H(z),\quad z\ge u'(0). \label{eq:3.2.8}
\end{align}
Под $u'(0)$ подразумевается правая производная. Чтобы получить последнее неравенство достаточно рассмотреть тестовую функцию $\varphi$, удовлетворяющую условиям $\varphi(0)=u(0)$, $\varphi'(0)=z>u'(0)$.

Результаты, собранные в следующей лемме, доказаны в \cite{Son86} (теоремы 3.3, 2.1, 2.2).
\begin{lemma} \label{lem:3.1}
Предположим, что множество $A$ компактно. Тогда функция Беллмана $v$ ограничена и равномерной непрерывна. Кроме того, $v$ является единственным CVS уравнения HJB (\ref{eq:3.2.6}) в классе ограниченных равномерно непрерывных функций.
\end{lemma}
Заметим, что предположение (A3) работы \cite{Son86}, касающееся существования <<внутреннего направления>>, выполняется, так как $\sup\{\alpha-q:\alpha\in A, q\in Q\}>0$ в силу (\ref{eq:3.2.2}).

Обозначим через $\mathscr M_H=\arg\min_{z\in\mathbb R} H(z)$ множество точек минимума $H$.
\begin{lemma} \label{lem:3.2}
Множество $\mathscr M_H\subset\mathbb R_+$ непусто, замкнуто и выпукло.
\end{lemma}
\begin{proof}
Множество $\mathscr M_H$ является замкнутым и выпуклым вследствие непрерывности и выпуклости $H$. Для $z<0$ имеем
$$\widehat R(z)>\widehat R(0),\quad \widehat C(z)=-C(0)=\widehat C(0),$$
так как $R$ неотрицательна, $R(0)=0$, и $C$ --- неубывающая функция. Таким образом, $H$, суженная на $(-\infty,0]$, достигает строгого глобального минимума в точке $z=0$. Для $z>0$ имеем
$$\widehat R(z)\ge R(0),\quad \widehat C (z)\ge \alpha z-C(\alpha)$$
для любого $\alpha>0$, $\alpha\in A$. Таким образом, $H(z)\to +\infty$, $z\to +\infty$. Отсюда следует, что $\emptyset\neq\mathscr M_H\subset\mathbb R_+$.
\end{proof}

Пусть $I$ --- интервал (т.е. выпуклое множество) в $\mathbb R$. Напомним, что функция $\psi:I\mapsto\mathbb R$, называется абсолютно непрерывной, если для любого $\varepsilon>0$ существует $\delta>0$ такое, что
$$ \sum_{i=1}^k |\psi(b_i)-\psi(a_i)|<\varepsilon $$
для любых дизъюнктных интервалов $(a_i,b_i)$, $i=1,\dots,k$ таких, что $[a_i,b_i]\subset I$, и
$ \sum_{i=1}^k (b_i-a_i)<\delta. $
Множество абсолютно непрерывных функций обозначим $AC(I)$. Функция $\psi:I\mapsto\mathbb R$ называется локально абсолютно непрерывной, если $\psi \in AC([a,b])$ для любого $[a,b]\subset I$. Любая локально абсолютно непрерывная функция $\psi$ является почти всюду (п.в.) дифференцируемой, и может быть восстановлена по её производной с помощью интеграла Лебега (см., например, \cite[Theorem 3.30]{Leo09}):
$$ \psi(x)=\psi(\overline x)+\int_{\overline x}^x\psi'(y)\,dy,\quad \overline x, x\in I.$$
Когда мы пишем <<п.в>>, мы всегда имеем ввиду <<почти всюду относительно меры Лебега>>.

Будем использовать следующий хорошо известный результат (см. \cite{Nat64} (Глава IX, упражнение 13) или \cite[теорема 2]{Vill84}):
\begin{lemma} \label{lem:3.3}
Пусть $\psi:[a,b]\mapsto\mathbb R$ --- непрерывная и строго монотонная функция. Тогда $\psi^{-1}$ абсолютно непрерывна, если и только если $\psi'\neq 0$ п.в. на $(a,b)$.
\end{lemma}

Обозначим через $\zeta=\min\mathscr M_H\ge 0$ наименьшую точку минимума функции $H$.
\begin{theorem} \label{th:3.1}
Функция Беллмана $v$ ограничена:
$$ v(x)\le \frac{H(0)}{\beta}=\lim_{y\to\infty} v(y)$$
и допускает следующее представление:
\begin{itemize}
\item[(i)] если $\zeta=0$, то
\begin{equation} \label{eq:3.2.9}
v(x)=H(0)/\beta,
\end{equation}
\item[(ii)] если $\zeta>0$ то
\begin{equation} \label{eq:3.2.10}
v(x)=\frac{H(\xi(x))}{\beta}=\frac{H(\zeta)}{\beta}+\int_0^x\xi(y)\,dy,
\end{equation}
где $\xi(x)$ определяется уравнением
$$x=\Psi(\xi):=-\int_\xi^{\zeta}\frac{H'(z)}{\beta z}\,dz,\quad \xi\in (0,\zeta],\quad x\ge 0.$$
\end{itemize}
В случае (ii) $v$ является строго возрастающей и строго вогнутой. Кроме того, $v'$ абсолютно непрерывна и удовлетворяет условиям $v'(0)=\zeta$, $\lim_{x\to\infty} v'(x)=0.$ Наконец, $v''<0$ п.в.
\end{theorem}
\begin{proof}
Проверим сначала, что (\ref{eq:3.2.9}), (\ref{eq:3.2.10}) являются CVS уравнения (\ref{eq:3.2.6}). Если $\zeta=0$, то (\ref{eq:3.2.9}) удовлетворяет (\ref{eq:3.2.7}), (\ref{eq:3.2.8}). Следовательно, (\ref{eq:3.2.9}) является CVS уравнения (\ref{eq:3.2.6}).

Предположим что $\zeta>0$. Поскольку $H$ выпукла, то её производная $H'$ существует на множестве $G=(0,\zeta)\backslash D$, где $D$ не более чем счётно, и $H'$ неубывающая на $G$ функция: см. \cite{Roc70} (теорема 25.3). Далее, $H'(x)\le 0$, $x\in G$, так как $H$ убывает на $(0,\zeta)$. Если $H'(x)=0$ для некоторого $x\in G$, то $x<\zeta$ является точкой минимума функции $H$. Следовательно, $H'(x)<0$, $x\in G$.

Рассмотрим следующую непрерывную строго убывающую функцию
$$ \Psi(\xi)=-\int_\xi^{\zeta}\frac{H'(z)}{\beta z}\,dz,\quad \xi\in (0,\zeta].$$
Имеем $\Psi(\zeta)=0$,
$$ \Psi(0+)=-\lim_{\xi\searrow 0} \int_\xi^{\zeta}\frac{H'(s)}{\beta s}\,ds=+\infty,$$
так как $H'$ п.в. ограничена сверху отрицательной константой в правой окрестности $0$.
Отсюда следует, что формула
$$ x=\Psi(\xi)$$
корректно определяет обратную функцию $\xi=\Psi^{-1}:\mathbb R_+\mapsto (0,\zeta]$, которая является строго убывающей и непрерывной.

Кроме того, так как $\Psi'(\xi)<0$ п.в. на  $(0,\zeta)$, функция $\xi(x)=\Psi^{-1}(x)$ является локально  абсолютно непрерывной на $\mathbb R_+$ по лемме \ref{lem:3.3}. Но так как $\xi$ монотонна и ограниченна, то $\xi\in AC(\mathbb R_+)$.
По лемме \ref{lem:3.3} заключаем также, что $\xi'(x)<0$ п.в. на $(0,\infty)$, так как $\Psi=\xi^{-1}$ локально  абсолютно непрерывна на $(0,\zeta]$.

Далее, по следствию 3.50 из \cite{Leo09}, можно применить правило дифференцирования сложной функции:
\begin{equation} \label{eq:3.2.11}
 1=\frac{d}{dx}\Psi(\xi(x))=\frac{H'(\xi(x))}{\beta\xi(x)}\xi'(x)\quad \textrm{п.в. на } (0,\infty).
\end{equation}
Функция $H(\xi)$ абсолютно непрерывна как суперпозиция абсолютно непрерывной функции $\xi$ и непрерывной по Липшицу функции $H$ (напомним, что $H$ выпукла). Следовательно,
$$ H(\xi(x))=H(\zeta)+\int_0^x \frac{d}{dy}H(\xi(y))\,dy.$$
По правилу дифференцирования сложной функции и формуле (\ref{eq:3.2.11}) находим
$$ H(\xi(x))=H(\zeta)+\int_0^x\xi'(y) H'(\xi(y))\,dy=H(\zeta)+\beta\int_0^x \xi(y)\,dy.$$

Теперь легко видеть, что
$$ u(x)=\frac{H(\zeta)}{\beta}+\int_0^x\xi(y)\,dy=\frac{H(\xi(x))}{\beta}$$
является CVS уравнения (\ref{eq:3.2.6}). В самом деле, $u'(x)=\xi(x)$, $x>0$ и
$$\beta u(x)-H(u'(x))=\beta u(x)-H(\xi(x))=0,\quad x>0.$$
Граничное условие (\ref{eq:3.2.8}) выполняется в силу определения $\zeta$:
$$ \beta u(0)=H(\zeta)\le H(z)\quad \text{for all}\quad z.$$

Заметим, что $u$ является строго возрастающей, так как $u'=\xi>0$, и строго вогнутой, так как $\xi$ --- строго убывающая (см. \cite{HirUrrLem01}, глава B, теорема 4.1.4). Далее, если $\zeta>0$, то
$$ u(x)\le\lim_{y\to\infty} u(y)=\lim_{y\to\infty}\frac{H(\xi(y))}{\beta}=\frac{H(0)}{\beta}.$$
Другие свойства производных (\ref{eq:3.2.10}), упомянутые в формулировке теоремы \ref{th:3.1}, очевидным образом следуют из определения функции $\xi$.

Мы доказали что  формулы (\ref{eq:3.2.9}), (\ref{eq:3.2.10}) определяют некоторое CVS уравнения (\ref{eq:3.2.6}). Если $A$ компактно, то (\ref{eq:3.2.9}), (\ref{eq:3.2.10}) определяют функцию Беллмана (\ref{eq:3.2.4}) в силу результата о единственности, сформулированного в лемме \ref{lem:3.1}.

В общем случае положим $A_c=A\cap [0, c]$, и обозначим через $H_c$, $v_c$ соответствующие гамильтониан и функцию Беллмана.
Для $z\le 0$ имеем $H_c(z)=H(z)=-C(0)$. Пусть
$$ \widehat\alpha(z)\in\arg\max_{a\in A}\{z\alpha-C(\alpha)\},\quad z>0. $$
Если $\widehat\alpha(z)>0$, то неравенство
$$-C(0)\le \widehat\alpha(z) \left(z -\frac{C(\widehat\alpha(z))}{\widehat\alpha(z)}\right)$$
и условие коэрцитивности (\ref{eq:3.2.3}) гарантируют, что $\widehat\alpha(z)$, $z\in [0,\overline z]$ ограничена сверху для любого фиксированного $\overline z>0$.
Таким образом, для любого $\overline z>0$ существует $\overline c>0$ такое, что
$$ \sup_{a\in A}\{z\alpha-C(\alpha)\}=\sup_{a\in A_c}\{z\alpha-C(\alpha)\}$$
и $H_c(z)=H(z)$ для $|z|\le \overline z$, $c\ge\overline c$. Полагая $\overline z>\zeta$, в силу выпуклости функции $H$ заключаем, что $\zeta$ является наименьшей точкой минимума $H_c$ при достаточно большом $c$.

Поскольку выражения (\ref{eq:3.2.9}), (\ref{eq:3.2.10}) зависят только от значений гамильтониана на отрезке $[0,\zeta]$, то они определяют CVS уравнения (\ref{eq:3.2.6}) с гамильтонианами $H$ и $H_c$ для $c\ge\overline c$. Но по лемме \ref{lem:3.1} $v_c$ является единственным CVS уравнения (\ref{eq:3.2.6}) с гамильтонианом $H_c$. Отсюда следует, что функции (\ref{eq:3.2.9}), (\ref{eq:3.2.10}) совпадают с $v_c$, $c\ge\overline c$.

Ясно, что $v_c\le v$. Остается доказать обратное неравенство. Для любой допустимой стратегии $(\alpha,q)\in\mathscr A(x)$ имеем
\begin{equation} \label{eq:3.2.12}
\frac{d}{dt}(e^{-\beta t}v_c(X_t))=e^{-\beta t}(-\beta v_c(X_t)+(\alpha_t-q_t)v_c'(X_t))\quad \text{п.в.},
\end{equation}
где $X$ определяется (\ref{eq:3.2.1}). Из уравнения HJB (\ref{eq:3.2.7}) находим
\begin{equation} \label{eq:3.2.13}
 \beta v_c(X_t)\ge R(q_t)-q_t v_c'(X_t)+\alpha_t v_c'(X_t)-C(\alpha_t)\quad \text{п.в.}
\end{equation}
Заметим, что равенство (\ref{eq:3.2.7}) справедливо при $x=0$ в силу свойства непрерывности. Из (\ref{eq:3.2.12}), (\ref{eq:3.2.13}) получаем неравенство
$$ -\frac{d}{dt}(e^{-\beta t}v_c(X_t))\ge e^{-\beta t} (R(q_t)-C(\alpha_t)) \quad \text{п.в.} $$
Отсюда следует что
$$ v_c(x)-e^{-\beta T}v_c(X_T)\ge\int_0^T e^{-\beta t} (R(q_t)-C(\alpha_t))\, dt$$
для любого $T>0$. Так как $v_c$ ограничена, то
$$ v_c(x)\ge \int_0^\infty e^{-\beta t} (R(q_t)-C(\alpha_t)),\quad (\alpha,q)\in\mathscr A(x).$$
Таким образом, $v_c\ge v$.
\end{proof}

В ходе доказательства было установлено что, переход от множества $A$ к $A\cap [0,c]$ не влияет на функцию Беллмана $v$ при достаточно больших $c$.

Заметим также, что если $\zeta>0$, то оптимальная дисконтированная прибыль не превосходит $H(0)/\beta$. Если $\zeta=0$, то дисконтированная прибыль $H(0)/\beta$ может быть получена при нулевом начальном запасе. Более того, любой начальный запас $x>0$ бесполезен.

Для функции $f:\mathbb R\mapsto (-\infty,+\infty]$ обозначим через $f^*:\mathbb R\mapsto (-\infty,+\infty]$ её преобразование Юнга-Фенхеля:
$$ f^*(z)=\sup_{x\in\mathbb R}\{zx-f(x)\},$$
и через $\co f$ --- её выпуклую оболочку:
$$ (\co f)(x)=\inf\{\delta f(x_1)+(1-\delta) f(x_2):\delta\in [0,1],\ x_i\in\dom f,\ \delta x_1+(1-\delta) x_2=x\},$$
где $\dom f=\{x:f(x)<\infty\}$. Для $G\subset\mathbb R$ обозначим через $\co G$ пересечение всех интервалов, содержащих $G$.
Следующий результат содержится в работе \cite{HirUrrLem93} (глава X, предложение 1.5.4).

\begin{lemma} \label{lem:3.4}
Пусть $G\subset\mathbb R$ --- непустое замкнутое множество и $f:G\mapsto\mathbb R$ --- непрерывная функция. Пусть $f(x)=+\infty$, $x\not \in G$ и предположим, что $f$ $1$-коэрцитивна: $f(x)/|x|\to+\infty$, $|x|\to\infty$. Тогда
$$\co f=f^{**},\quad \dom (\co f)=\co G,$$
и для любого $x\in\co G$ существуют $x_1, x_2\in G$ и $\delta\in (0,1)$ такие, что
\begin{equation} \label{eq:3.2.14}
x=\delta x_1+(1-\delta) x_2,\quad (\co f)(x)=\delta f(x_1)+(1-\delta) f(x_2).
\end{equation}
\end{lemma}

Функции $C$, $-R$ удовлетворяют условиям леммы \ref{lem:3.4}. Пусть $C(\alpha)=+\infty$, $\alpha\not\in A$ и $R(q)=-\infty$, $q\not\in Q$. Для унификации системы обозначений, обозначим через
$$\widetilde C=C^{**}=\co C$$
замкнутую выпуклую оболочку функции $C$ и через
$$\widetilde R=-(-R)^{**}=-\co(-R)$$
--- замкнутую вогнутую оболочку функции $R$.

Сравнивая данные обозначения с предыдущими:
%\begin{equation} \label{eq:3.2.15}
$$\widehat C(z)=C^*(z)=\sup_{x\in\mathbb R}\{xz-C(x)\}=C^{***}(z)=\sup_{x\in\mathbb R}\{xz-\widetilde C(x)\},$$
%\end{equation}
\begin{align} \label{eq:3.2.15}
\widehat R(z)=\sup_{x\in\mathbb R}\{R(x)-xz\}=\sup_{x\in\mathbb R}\{x\cdot (-z)-(-R(x))\}=(-R)^*(-z)\nonumber\\
=(-R)^{***}(-z)=\sup_{x\in\mathbb R}(x\cdot (-z)-(-R)^{**}(x)\}=\sup_{x\in\mathbb R}\{\widetilde R(x)-xz\},
\end{align}
заключаем, что гамильтониан (\ref{eq:3.2.5}) может быть представлен следующим образом:
\begin{equation} \label{eq:3.2.16}
 H(z)=C^*(z)+(-R)^*(-z)=\sup_{x\in\mathbb R}\{xz-\widetilde C(x)\}+\sup_{x\in\mathbb R}\{\widetilde R(x)-xz\}.
\end{equation}

Введем \emph{овыпукленную задачу}:
\begin{equation} \label{eq:3.2.17}
\widetilde v(x)=\sup_{(\alpha,q)\in\widetilde{\mathscr A}(x)}\int_0^\infty e^{-\beta t}(\widetilde R(q_t)-\widetilde C(\alpha_t))\,dt,
\end{equation}
где $\widetilde{\mathscr A}(x)$ --- множество измеримых по Борелю функций $\alpha:\mathbb R_+\mapsto\co A$, $q:\mathbb R_+\mapsto \co Q$ таких, что $X_t^{x,\alpha,q}\ge 0$. Заметим, что $\widetilde C$ по-прежнему удовлетворяет условию (\ref{eq:3.2.3}): см. \cite[глава E, предложение 1.3.9(ii)]{HirUrrLem01}. Ясно, что $v\le\widetilde v$. Но, так как гамильтониан для овыпукленной задачи такой же, как и для исходной (см. (\ref{eq:3.2.16})), то по теореме \ref{th:3.1} имеем $\widetilde v=v$.

Расширим класс стратегий производства и назначения цены. \emph{Распределенные управления} $q_t(dy)$ и $\alpha_t(dy)$ представляют собой отображения отрезка $[0,\infty)$ в множества вероятностных мер на $Q$ и $A$ такие, что функции
$$ t\mapsto\int_Q\varphi(y)\,q_t(dy),\qquad t\mapsto\int_A\varphi(y)\,\alpha_t(dy)$$
измеримы по Борелю для любой непрерывной функции $\varphi$. Динамика движения товарного запаса при использовании распределенных управлений определяется следующим образом
$$ X_t=x+\int_0^t\int_Q y\,q_s(dy)ds-\int_0^t\int_A y\,\alpha_s(dy)ds.$$
Класс $\mathscr A_r(x)$ допустимых распределенных стратегий содержит лишь те, которые удерживают $X_t$ в неотрицательной области. Соответствующая функция Беллмана определяется следующим образом:
\begin{equation} \label{eq:3.2.18}
v_r(x)=\sup_{(\alpha,q)\in\mathscr A_r(x)}\left(\int_0^\infty e^{-\beta t}\int_Q R(y)\,q_t(dy)dt -\int_0^\infty e^{-\beta t} \int_A C(y)\,\alpha_t(dy)dt\right).
\end{equation}
Задачу (\ref{eq:3.2.18}) будем называть \emph{ослабленной}.


Заметим, что допустимая распределенная стратегия $(\alpha_t(dx),q_t(dx))$ индуцирует допустимую обычную стратегию
$$\left(\int_A x\,\alpha_s(dx),\int_Q x\, q_s(dx)\right)\in(\co A,\co Q)$$
для овыпукленной задачи (\ref{eq:3.2.17}). Следовательно, из неравенство Иенсена вытекает, что
\begin{align*}
&\int_0^\infty e^{-\beta s}\left(\int_Q R(x)\, q_s(dx)-\int_A C(x)\,\alpha_s(dx)\right)\,ds\\
&\le \int_0^\infty e^{-\beta s}\left(R\left(\int_Q x\, q_s(dx)\right)-C\left(\int_A x\,\alpha_s(dx)\right)\right)\,ds\le\widetilde v(x),
\end{align*}
так как $\widetilde C\le C$, $\widetilde R\ge R$. Таким образом, $v_r\le\widetilde v$, и очевидное неравенство $v\le v_r$ приводит к следующему результату.
\begin{theorem} \label{th:3.2}
Функции Беллмана (\ref{eq:3.2.4}), (\ref{eq:3.2.17}), (\ref{eq:3.2.18}) соответствующие исходной, овыпукленной и ослабленной задачам совпадают:
$v=v_r=\widetilde v.$
\end{theorem}

Равенство $v=v_r$ для задачи с фазовыми ограничениями в случае компактных множеств состояний и управлений было установлено в \cite{Lor87}.

\section{Оптимальные стратегии в случае нулевого начального запаса} \label{sec:3.3}
%\setcounter{equation}{0}
В этом разделе мы рассмотрим случай нулевого начального запаса: $X_0=0$. Для любой константы $\widehat u\in Q\cap A$ \emph{статическая стратегия} $\alpha_t=q_t= \widehat u$ допустима. Если она оптимальна, то
\begin{equation} \label{eq:3.3.1}
\widehat u\in\mathscr M:=\arg\max_{u\in Q\cap A}\{R(u)-C(u)\}.
\end{equation}
Для $\eta\in\mathbb R$ положим
$$\mathscr M_R(\eta)=\arg\max_{q\in Q}\{R(q)-\eta q\},\qquad
  \mathscr M_C(\eta)=\arg\max_{\alpha\in A}\{\alpha\eta-C(\alpha)\},$$
и $\mathscr M_\eta=\mathscr M_R(\eta)\cap\mathscr M_C(\eta)$. Напомним, что $\zeta$ --- наименьшая точка минимума $H$.

\begin{theorem} \label{th:3.3}
Следующие условия эквивалентны.
\begin{itemize}
\item[(i)] Статическая стратегия $\alpha_t=q_t= \widehat u\in\mathscr M$ является оптимальной.
\item[(ii)]  $\mathscr M_\eta\neq\emptyset$ для некоторого $\eta\in\mathbb R$.
\item[(iii)]  $\mathscr M_\zeta\neq\emptyset$.
\end{itemize}
%Assume that $R\not\equiv 0$ and $C$ is strictly increasing. If a stationary strategy is not optimal, then there is no optimal strategy.
Если $\mathscr M_\eta\neq\emptyset$, то $\eta$ является точкой минимума функции $H$ и $\mathscr M_\eta=\mathscr M$.
\end{theorem}
\begin{proof}
По теореме \ref{th:3.1} имеем
\begin{equation} \label{eq:3.3.3}
\beta v(0)=H(\zeta)\le H(\eta)=\sup_{q\in Q}\{R(q)-\eta q\}+\sup_{\alpha\in A}\{\alpha\eta-C(\alpha)\},\quad \eta\in\mathbb R.
\end{equation}

(ii) $\Longrightarrow$ (i). Для любого $\widehat u\in\mathscr M_\eta$ из (\ref{eq:3.3.3}) находим
%If $\mathscr M_\eta\neq\emptyset$, then, by Lemma \ref{lem:3.4}, for any $\widehat u\in\mathscr M=\mathscr M_R(\eta)\cap\mathscr M_C(\eta)$ we have
\begin{equation} \label{eq:3.3.4}
v(0)\le (R(\widehat u)-C(\widehat u))/\beta=\int_0^\infty e^{-\beta t}(R(\widehat u)-C(\widehat u))\,dt.
\end{equation}
Таким образом, $\alpha_t=q_t= \widehat u$ является оптимальной стратегией и $\widehat u\in\mathscr M$.

(i) $\Longrightarrow$ (iii). Если статическая стратегия $\alpha_t=q_t= \widehat u$ оптимальна, то
$$ R(\widehat u)-C(\widehat u)=\beta\int_0^\infty e^{-\beta t}(R(\widehat u)-C(\widehat u))\,dt=\beta v(0)=H(\zeta).$$
Если $\widehat u\not\in\mathscr M_\zeta$, то получаем противоречие:
$$ H(\zeta)=\widehat R(\zeta)+\widehat C(\zeta)>R(\widehat u)-\zeta\widehat u+\zeta\widehat u-C(\widehat u)=R(\widehat u)-C(\widehat u).$$

(iii) $\Longrightarrow$ (ii) очевидно.

Пусть $\mathscr M_\eta\neq\emptyset$. Заметим, что для $\widehat u\in\mathscr M_\eta$ справедливо $H(\eta)=R(\widehat u)-C(\widehat u)$.
Отсюда следует, что неравенство в (\ref{eq:3.3.3}) не может быть строгим, так как из этого вытекало бы строгое неравенство в (\ref{eq:3.3.4}). Таким образом, если $\mathscr M_\eta\neq\emptyset$, то $H(\eta)=H(\zeta)$, и $\eta$ является точкой минимума функции $H$.

Пусть $\overline u\in\mathscr M_\eta$. Тогда
\begin{equation} \label{eq:3.3.5}
 R(u)-\eta u\le R(\overline u)-\eta \overline u,\quad \eta u-C(u)\le \eta\overline u-C(\overline u)
\end{equation}
для любого $u\in Q\cap A$. После суммирования получаем
$$ R(u)-C(u)\le R(\overline u)-C(\overline u).$$
Следовательно, $\overline u\in\mathscr M$.

Теперь возьмем некоторое $u\in\mathscr M$. Если $u\not\in\mathscr M_\eta$, то по крайней мере одно из неравенств (\ref{eq:3.3.1}) является строгим, и после суммирования получаем противоречие с определением $u$:
$$  R(u)-C(u)< R(\overline u)-C(\overline u).$$
Таким образом, оба неравенства (\ref{eq:3.3.5}) являются, фактически, равенствами, и $u\in M_\eta$ вместе с $\overline u$.
\end{proof}

Чтобы дать экономическую интерпретацию условия $\mathscr M_\eta\neq\emptyset$, предположим что в некоторой фирме управление производственным цехом и отделом продаж происходит независимо друг от друга. Производственный цех продает товар отделу продаж  по некоторой \emph{теневой} цене $\eta$ и получает мгновенный доход $\eta\alpha-C(\alpha)$. Отдел продаж получает мгновенный доход $R(q)-\eta q$ продавая товар на рынке. Для каждой возможной цены $\eta$ отделы пытаются выбрать оптимальные стратегии $\widehat\alpha$, $\widehat q$. Равновесие $\widehat\alpha=\widehat q$  соответствует теневой цене $\eta$ такой, что $\mathscr M_\eta\neq\emptyset$. По теореме \ref{th:3.2} статическая стратегия является оптимальной в точности тогда, когда данное равновесие существует.

Аналогично равновесию механической системы, равновесная теневая цена $\eta$ может быть определена как точка минимума гамильтониана $H$. По теореме \ref{th:3.1} наименьшая теневая цена $\zeta$ совпадает с предельной непрямой полезностью $v'(0)$ нулевого запаса.

Для функции $f:\mathbb R\mapsto (-\infty,+\infty]$ обозначим через $\partial f$ её субдифференциал в точке $z$:
$$ \partial f(z)=\{\gamma\in\mathbb R: f(x)\ge f(z)+\gamma (x-z),\ x\in\mathbb R\}.$$
Для любой полунепрерывной снизу выпуклой функции $f\not\equiv +\infty$ верно (см. \cite[теорема 23.5]{Roc70} или \cite[предложение 11.3]{RockWets09}), что
\begin{equation} \label{eq:3.3.6}
\arg\max_{x\in\mathbb R}\{zx-f(x)\}=\partial f^*(z).
\end{equation}

\begin{theorem} \label{th:3.4}
Предположим что множества $Q$, $A$ выпуклы, функция $R$ вогнута и функция $C$ выпукла.
Тогда для любого $\eta\in\mathscr M_H$ имеем
$\mathscr M_\eta\neq\emptyset$.
Следовательно, стационарная стратегия является оптимальной.
\end{theorem}
\begin{proof}
Из определения $R$, $C$ вытекает, что каждое из множеств $\mathscr M_R(\eta)$, $\mathscr M_C(\eta)$ непусто при любом $\eta$.
Если $g(x)=f(-x)$, то $\partial g(x)=-\partial f(-x)$.
Используя формулу Моро-Рокафеллара (см. \cite[теорема 23.8]{Roc70}  или \cite[теорема 3.6.3]{AusTeb03}), из (\ref{eq:3.2.16}) находим
%Using the Moreau-Rockafellar formula (see \cite[Theorem 23.8]{Roc70}  or \cite[Theorem 3.6.3]{AusTeb03}), from (\ref{eq:3.2.16}) we get
\begin{equation} \label{eq:3.3.7}
0\in\partial H(\eta)=\partial C^*(\eta)-\partial(-R)^*(-\eta)=\{x-y:x\in\partial C^*(\eta), y\in\partial(-R)^*(-\eta)\}
\end{equation}
в точке $\eta$ минимума функции $H$. Более того, в силу (\ref{eq:3.3.6}) имеем
$$\partial(-R)^*(-\eta)=\arg\max_{x\in\mathbb R}\{-x\eta-(-R(x))\}=\arg\max_{x\in\mathbb R} \{R(x)-\eta x\}
=\mathscr M_R(\eta),$$
$$\partial C^*(\eta)=\arg\max_{x\in\mathbb R}\{x\eta -C(x)\}=\mathscr M_C(\eta).$$
Следовательно, соотношение (\ref{eq:3.3.7}) эквивалентно условию $0\in\mathscr M_C(\eta)-\mathscr M_R(\eta)$, и
$\mathscr M_\eta=\mathscr M_C(\eta)\cap\mathscr M_R(\eta)\neq\emptyset.$
\end{proof}

Для выпуклой задачи, рассмотренной в теореме \ref{th:3.4}, оптимальность стационарной стратегии более прямым методом была доказана в \cite{ChaJouTah03} (предложение 1).

Статическая стратегия, максимизирующая мгновенную прибыль, является стратегией $\widehat q$, максимизирующей значение вогнутой функции дохода $R(q)$ на $[0,\overline q\wedge\overline\alpha]$, где $\overline q\wedge\overline\alpha=\min\{\overline q,\overline\alpha\}$.
Очевидно, что $\widehat q<\overline\alpha$, если $\overline q<\overline\alpha$. Однако, в теореме \ref{th:3.3} утверждается что $\widehat\alpha=\widehat q$ является оптимальным.

Чтобы дать простую иллюстрацию данного результата рассмотрим случай постоянных затрат:
$C(\alpha)=c>0$, $\alpha\in [0,\overline\alpha].$
Кажется естественным, что только максимальная интенсивность производства $\alpha_t=\overline\alpha$ является оптимальной в данном случае, так как, в сравнении с другими производственными стратегиями, мы получаем дополнительный товар даром. Пусть $Q=[0,\overline q]$, $\overline q>0$. Статическая стратегия, максимизирующая мгновенный доход совпадает со стратегией $\widehat q$, максимизирующей вогнутую функцию прибыли $R(q)$ на $[0,\overline q\wedge\overline\alpha]$, где $\overline q\wedge\overline\alpha=\min\{\overline q,\overline\alpha\}$. Ясно, что $\widehat q<\overline\alpha$, если $\overline q<\overline\alpha$. Однако, теорема \ref{th:3.3} утверждает, что $\widehat\alpha=\widehat q$ оптимальны.

Объяснение данного <<парадокса>> состоит в том, что фактически оптимальная скорость продажи $\widehat q$ покрывается  любой скоростью производства $\alpha\ge\widehat q$. Дополнительная продукция, произведенная при использовании максимальной интенсивности $\overline\alpha$ не продается и остается неиспользованной.

Для невыпуклой задачи возможно, что $\mathscr M_\zeta=\emptyset$ и не существует оптимальной стационарной стратегии вида (\ref{eq:3.3.1}).
Конкретный пример будет рассмотрен в разделе \ref{sec:3.6}. Наша следующая цель состоит в описание оптимальных стратегий в этом случае.

По теоремам \ref{th:3.3}, \ref{th:3.4} для овыпукленной задачи (\ref{eq:3.2.17}) стационарная стратегия $\widetilde\alpha_t=\widetilde q_t=\widetilde u$,
\begin{equation} \label{eq:3.3.8}
\widetilde u\in\arg\max\{\widetilde R(u)-\widetilde C(u):u\in\co Q\cap\co A\},
\end{equation}
оптимальна для нулевого начального запаса, и
$$\widetilde v(0)=\frac{\widetilde R(\widetilde u)-\widetilde C(\widetilde u)}{\beta}.$$
Далее, по лемме 4 существуют $\gamma\in (0,1)$, $\nu\in (0,1)$, $q^i\in Q$, $\alpha^i\in A$, $i=1,2$ такие, что
\begin{equation} \label{eq:3.3.9}
\widetilde u=\gamma q^1+(1-\gamma) q^2=\nu\alpha^1+(1-\nu)\alpha^2,
\end{equation}
\begin{equation} \label{eq:3.3.10}
\widetilde R(\widetilde u)=\gamma R(q^1)+(1-\gamma) R(q^2),\quad \widetilde C(\widetilde u)=\nu C(\alpha^1)+(1-\nu) C(\alpha^2).
\end{equation}
Рассмотрим распределенные управления
\begin{equation} \label{eq:3.3.11}
 \overline q_t(dx)=\gamma\delta_{q^1}(dx)+(1-\gamma)\delta_{q^2}(dx),\quad
 \overline \alpha_t(dx)=\nu\delta_{\alpha^1}(dx)+(1-\nu)\delta_{\alpha^2}(dx),
\end{equation}
где $\delta_a$ --- мера Дирака, сконцентрированная в точке $a$. Эти управления допустимы для ослабленной задачи (\ref{eq:3.2.18}), так как
$$ \int_0^t \int_Q x\overline q_s(dx)\,ds-\int_0^t \int_A x\overline\alpha_s(dx)\,ds=t(\gamma q^1+(1-\gamma) q^2-\nu\alpha^1-(1-\nu)\alpha^2)=0.$$
На основании теоремы \ref{th:3.2} следующее простое вычисление показывает что стратегия (\ref{eq:3.3.11}) является оптимальной:
\begin{align*}
v_r(0)\ge &\int_0^\infty e^{-\beta s}\left(\int_Q R(x)\,\overline q_s(dx)-\int_A C(x)\,\overline \alpha_s(dx)\right)\,ds\nonumber\\
=&\int_0^\infty e^{-\beta s}\left(\widetilde R(\widetilde u)-\widetilde C(\widetilde u)\right)\,ds=\widetilde v(0).
\end{align*}

Таким образом, получаем следующий результат.
\begin{theorem} \label{th:3.5}
(i) Пусть $\widetilde C$, $\widetilde R$ --- выпуклая и вогнутая оболочки $C$, $R$. Тогда $\widetilde u$, определенная в (\ref{eq:3.3.8}), является оптимальной статической стратегией для овыпукленной задачи (\ref{eq:3.2.17}) с нулевым начальным запасом.

(ii) Существуют $q^i\in Q$, $\alpha^i\in A$, $i=1,2$ и $\gamma\in (0,1)$, $\nu\in (0,1)$ такие что (\ref{eq:3.3.9}), (\ref{eq:3.3.10}) справедливы. Распределенная статическая стратегия (\ref{eq:3.3.11}) дает решение задачи (\ref{eq:3.2.18}) с нулевым начальным запасом.
\end{theorem}

Стратегия (\ref{eq:3.3.11}) распределяет цену (соотв., интесивность производства) между двумя уровнями, которые определяются значением $\widehat u$, вычисленным для овыпукленной задачи, и положением  $R$ (соот., $C$) относительно её вогнутой (соотв., выпуклой) оболочки.

Рандомизация стратегий производства и продажи товаров, подразумеваемая распределенными управлениями, вряд ли может быть реализована практически. Поэтому, имеет смысл построить обычную приближенно оптимальную стратегию $(\alpha^\varepsilon,q^\varepsilon)\in\mathscr A(0)$.
Мы будем использовать следующее уточнение леммы \ref{lem:3.4}: см. \cite{HirUrrLem93} (глава X, теоремы 1.5.5, 1.5.6).
\begin{lemma} \label{lem:3.5}
Пусть предположения леммы \ref{lem:3.4} выполняются. Если для данного $x\in\co G$ точки $x_1$, $x_2$ в (\ref{eq:3.2.14}) различны, то $\co f$ аффинна на $(x_1,x_2)$ и, в дополнение к (\ref{eq:3.2.14}), справедливы равенства
$$ (\co f)(x_i)=f(x_i),\quad s x-(\co f)(x)=s x_i-f(x_i),\quad s\in\partial(\co f)(x),\quad i=1,2.$$
\end{lemma}

Для $\widetilde u$, определенного в (\ref{eq:3.3.8}), по теоремам \ref{th:3.3}, \ref{th:3.4} имеем
$$ \widetilde u\in\arg\max_{q\in \co Q} \{\widetilde R(q)-q\zeta\},\quad
   \widetilde u\in\arg\max_{\alpha\in\co A}\{\alpha\zeta-\widetilde C(\alpha)\}.$$
Отсюда следует что $\zeta\in\partial\widetilde C(\widetilde u)$ и, по лемме \ref{lem:3.5}, существуют $\nu\in (0,1)$, $\alpha^i\in A$, $i=1,2$, удовлетворяющие (\ref{eq:3.3.9}), такие, что
\begin{equation} \label{eq:3.3.12}
 \alpha^i\zeta-C(\alpha^i)=\widetilde u\zeta-\widetilde C(\widetilde u),\quad i=1,2.
\end{equation}
Аналогично, существуют $\gamma\in (0,1)$, $q^i\in Q$, $i=1,2$, удовлетворяющие (\ref{eq:3.3.9}), такие, что
\begin{equation} \label{eq:3.3.13}
 R(q^i)-q^i\zeta=\widetilde R(\widetilde u)-\widetilde u\zeta,\quad i=1,2.
\end{equation}
Из (\ref{eq:3.3.12}), (\ref{eq:3.3.13}) для любого $\varkappa\in [0,1]$ получаем
\begin{align} \label{eq:3.3.14}
\widetilde R(\widetilde u)-\widetilde C(\widetilde u) &= \varkappa (R(q^1)-C(\alpha^2))+(1-\varkappa)(R(q^2)-C(\alpha^1))\nonumber\\
 &+(\varkappa(\alpha^2-q^1)+(1-\varkappa)(\alpha^1-q^2))\zeta.
\end{align}

Можно считать, что $\alpha^2\ge\alpha^1$, $q^2\ge q^1$ и либо $\alpha^2>\alpha^1$, либо $q^2>q^1$, так как иначе $\alpha^1=\alpha^2=q^1=q^2=\widetilde u$, и обычная стационарная стратегия $\alpha_t=q_t=\widetilde u$ является оптимальной. Положим $\tau_i=\varepsilon i$, $i\in\mathbb Z_+=\{0,1,\dots\}$ и
\begin{equation} \label{eq:3.3.15}
\alpha_t^\varepsilon=\sum_{i=0}^\infty\left( \alpha^2 I_{[\tau_i,\tau_i+\varkappa\varepsilon)}(t)+\alpha^1 I_{[\tau_i+\varkappa\varepsilon,\tau_{i+1})}(t)\right),
\end{equation}
\begin{equation} \label{eq:3.3.16}
q^\varepsilon_t=\sum_{i=0}^\infty \left( q^1 I_{[\tau_i,\tau_i+\varkappa\varepsilon)}(t)+q^2 I_{[\tau_i+\varkappa\varepsilon,\tau_{i+1})}(t)\right),
\end{equation}
$$ \varkappa=\frac{q^2-\alpha^1}{q^2-\alpha^1+\alpha^2-q^1}\in (0,1).$$
Функция $\int_{\tau_i}^t(\alpha_s^\varepsilon - q^\varepsilon_s)\,ds$ возрастает на $[\tau_i,\tau_i+\varkappa\varepsilon)$ и убывает на $[\tau_i+\varkappa\varepsilon,\tau_{i+1})$. Она неотрицательна на $[\tau_i,\tau_{i+1}]$, так как
$$\int_{\tau_i}^{\tau_{i+1}}(\alpha_s^\varepsilon - q^\varepsilon_s)\,ds=\varkappa\varepsilon(\alpha^2-q^1)+(1-\varkappa)\varepsilon(\alpha^1-q^2)=0$$
по определению $\varkappa$. Таким образом, $(\alpha^\varepsilon,q^\varepsilon)\in\mathscr A(0)$.

Далее,
\begin{align*}
 \int_{\tau_i}^{\tau_{i+1}}(R(q_t^\varepsilon)-C(\alpha_t^\varepsilon))\,dt &=\frac{R(q^1)-C(\alpha^2)}{\beta} \left(e^{-\beta\tau_i}-e^{-\beta(\tau_i+\varkappa\varepsilon)}\right)\\
 &+\frac{R(q^2)-C(\alpha^1)}{\beta} \left(e^{-\beta(\tau_i+\varkappa\varepsilon)}-e^{-\beta\tau_{i+1}}\right).
\end{align*}
Суммируя данные выражения, получаем
\begin{align*}
 &\lim_{\varepsilon\to 0}\int_0^\infty e^{-\beta t} (R(q_t^\varepsilon)-C(\alpha_t^\varepsilon))\,dt =\lim_{\varepsilon\to 0}\frac{1}{1-e^{-\beta\varepsilon}}  \biggl( \frac{R(q^1)-C(\alpha^2)}{\beta} \left(1-e^{-\beta\varkappa\varepsilon}\right)\nonumber\\
 &+\frac{R(q^2)-C(\alpha^1)}{\beta} \left(e^{-\beta\varkappa\varepsilon}-e^{-\beta\varepsilon}\right)\biggr)\nonumber\\
 &=\frac{R(q^1)-C(\alpha^2)}{\beta}\varkappa+\frac{R(q^2)-C(\alpha^1)}{\beta}(1-\varkappa)=\frac{\widetilde R(\widetilde u)-\widetilde C(\widetilde u)}{\beta}=\widetilde v(0),
\end{align*}
здесь в предпоследнем равенстве использовано (\ref{eq:3.3.14}).

Итак, стратегия (\ref{eq:3.3.15}), (\ref{eq:3.3.16}) является приближенно оптимальной:
$$ v(0)=\lim_{\varepsilon\to 0}\int_0^\infty e^{-\beta t}(R(q_t^\varepsilon)-C(\alpha_t^\varepsilon))\,dt. $$
Заметим также, что при этой стратегии уровень запаса $X_t$  демонстрирует циклическое поведение и $0\le X_t\le\varkappa(\alpha^2-q^1)\varepsilon$.

\section{Оптимальные стратегии в случае положительного начального запаса} \label{sec:3.4}
%\setcounter{equation}{0}
Полное описание оптимальных стратегий дано в теореме \ref{th:3.6}. В доказательстве этой теоремы мы используем следующий результат.
\begin{lemma} \label{lem:3.6}
Пусть $f$ удовлетворяет предположениям леммы \ref{lem:3.4}, и пусть $F$ --- ко-счетное множество, где функция $f^*$ является дифференцируемой.  С учетом (\ref{eq:3.3.6}) положим
$$ \{\widehat x(z)\}=\{(f^*)'(z)\}=\arg\max_{x\in\mathbb R}\{xz-\co f(x)\},\quad z\in F.$$
Тогда
\begin{equation} \label{eq:3.4.1}
\widehat x(z)\in\dom f,\quad \co f(\widehat x(z))=f(\widehat x(z))
\end{equation}
для $z\in F$.
\end{lemma}
\begin{proof}
Если (\ref{eq:3.4.1}) не выполняется, то из леммы \ref{lem:3.5} следует, что $\co f$ является аффинной в окрестности $\widehat x(z)$. Следовательно, множество $\arg\max_{x\in\mathbb R}\{xz-\co f(x)\}$ содержит эту окрестность, и $z\not\in F$.
\end{proof}

Если $\zeta=0$, то, по теореме \ref{th:3.1}, $v$ является константой. Этот случай в некотором смысле тривиален, так как оптимальная стратегия для нулевого начального запаса $X_0=0$ сохраняет это свойство и для $X_0>0$. Поэтому, мы будем предполагать что $\zeta>0$.

Выпуклые функции $\widehat R$, $\widehat C$ дифференцируемы ко-счетном подмножестве $F$ отрезка $(0,\zeta)$. Следовательно, в силу (\ref{eq:3.3.6}) мы заключаем, что каждое из множеств
$$ \widetilde{\mathscr M}_R(z)=\arg\max_{q\in \co Q} \{\widetilde R(q)-qz\},\quad
   \widetilde{\mathscr M}_C(z)=\arg\max_{\alpha\in\co A}\{\alpha z-\widetilde C(\alpha)\}$$
содержит ровно одну точку:
\begin{align}
\widetilde{\mathscr M}_R(z)&=\arg\max_{q\in \co Q}(\widetilde R(q)-zq\}=\arg\max_{x\in\mathbb R}(-zx-(-\widetilde R(x))=\partial (-\widetilde R)^*(-z)\nonumber\\
&=-\partial\widehat R(z)=\{-\widehat R'(z)\},\label{eq:3.4.2}\\
\widetilde{\mathscr M}_C(z)&=\arg\max_{\alpha\in \co A}(z\alpha-\widetilde C(\alpha)\}=\partial C^*(z)=\{\widehat C'(z)\}\nonumber
\end{align}
для $z\in F$. В (\ref{eq:3.4.2}) мы использовали равенство $\widehat R(z)=(-R)^*(-z)=(-\widetilde R)^*(-z)$: см. (\ref{eq:3.2.15}).

\begin{theorem} \label{th:3.6}
Пусть $F\subset (0,\zeta)$ --- ко-счетное множество, где выпуклые функции $\widehat R$, $\widehat C$ дифференцируемы. Положим
$$ \{\widehat q(z)\}=\arg\max_{q\in \co Q} \{\widetilde R(q)-q z\},\quad
   \{\widehat\alpha(z)\}=\arg\max_{\alpha\in\co A}\{\alpha z-\widetilde C(\alpha)\},\quad z\in F,$$
$$\widehat u\in\arg\max(\widetilde R(u)-\widetilde C(u):u\in\co Q\cap\co A).$$
Для заданного начального запаса $x>0$ положим
$$\tau=\frac{1}{\beta}\ln\frac{v'(0)}{v'(x)}$$
и определим $X$ уравнением
\begin{equation} \label{eq:3.4.3}
v'(X_t)=v'(x)e^{\beta t},\quad t\in [0,\tau].
\end{equation}

Далее, положим $\mathscr T=\{t\in [0,\tau]:v'(X_t)\in F\}$ и рассмотрим стратегию
\begin{equation} \label{eq:3.4.4}
\alpha^*_t=\widehat\alpha(v'(X_t)),\quad q^*_t=\widehat q(v'(X_t)),\quad t\in\mathscr T,
\end{equation}
\begin{equation} \label{eq:3.4.5}
 \alpha^*_t=q^*_t=\widehat u,\quad t>\tau.
\end{equation}
На счетном множестве $[0,\tau]\backslash \mathscr T$ значения $\alpha^*_t$, $q_t^*$ могут быть определены произвольным образом.

(i) Стратегия (\ref{eq:3.4.4}) является оптимальной для овыпукленной задачи (\ref{eq:3.2.17}).

(ii) Имеем
\begin{equation} \label{eq:3.4.6}
(\alpha_t^*,q_t^*)\in\dom C\times\dom R,\quad \widetilde C(\alpha^*_t)=C(\alpha^*_t),\quad \widetilde R(\alpha_t^*)=R(\alpha_t^*),\quad t\in\mathscr T.
\end{equation}

(iii) Существуют $q^i\in Q$, $\alpha^i\in A$, $i=1,2$ и $\gamma\in (0,1)$, $\nu\in (0,1)$ такие, что справедливы равенства (\ref{eq:3.3.9}), (\ref{eq:3.3.10}). Заменяя (\ref{eq:3.4.5}) статическим распределенным управлением
\begin{equation} \label{eq:3.4.7}
\overline q(dx)=\gamma\delta_{q^1}(dx)+(1-\gamma)\delta_{q^2}(dx),\quad
 \overline \alpha(dx)=\nu\delta_{\alpha^1}(dx)+(1-\nu)\delta_{\alpha^2}(dx),
\end{equation}
получаем решение ослабленной задачи (\ref{eq:3.2.18}).

(iv) Если $\mathscr M_\zeta\neq\emptyset$, то заменяя (\ref{eq:3.4.5}) на
\begin{equation} \label{eq:3.4.8}
\widehat u\in\arg\max_{u\in Q\cap A}(R(u)-C(u)),
\end{equation}
получаем оптимальное решение задачи (\ref{eq:3.2.4}).
\end{theorem}
\begin{proof}
Поскольку $H$ дифференцируема на $F$ и $v':(0,\infty)\mapsto (0,\zeta)$ является биекцией, то $H'(v'(x))$ корректно определена на ко-счетном множестве $(v')^{-1}(F)\subset (0,\infty)$. Вторая производная $v''$ существует п.в. на интервале $(0,\infty)$. По правилу дифференцирования сложной функции (см. \cite[следствие 3.48]{Leo09}) из уравнения HJB следует что
\begin{equation} \label{eq:3.4.9}
 \beta v'(x)=H'(v'(x))v''(x)\quad\textrm{п.в. на } (0,\infty).
\end{equation}

Положим $\widehat q(z)=-\widehat R'(z)$, $\widehat\alpha(z)=\widehat C'(z)$, $z\in F$. Тогда
\begin{equation} \label{eq:3.4.10}
 H'(z)=\widehat R'(z)+\widehat C'(z)=-\widehat q(z)+\widehat\alpha(z)<0,\quad z\in F.
\end{equation}
Формально используя $\widehat q(v'(x))$, $\widehat\alpha(v'(x))$ как \emph{управления с обратной связью}, с помощью формул (\ref{eq:3.4.9}), (\ref{eq:3.4.10}) получаем
$$ dt=\frac{d X_t}{\widehat\alpha(v'(X_t)-\widehat q(v'(X_t))}=\frac{dX_t}{H'(v'(X_t))}=\frac{1}{\beta}d (\ln v'(X_t)). $$
Используя начальное условие $X_0=x$, после интегрирования находим
\begin{equation} \label{eq:3.4.11}
 t=\frac{1}{\beta}\ln\frac{v'(X_t)}{v'(x)}.
\end{equation}
Функция
$$ y\mapsto\frac{1}{\beta}\ln\frac{v'(y)}{v'(x)}:[0,x]\mapsto [0,\tau]$$
является биективной. Определим функцию $t\mapsto X_t:[0,\tau]\mapsto[0,x]$ уравнением (\ref{eq:3.4.3}), которое идентично (\ref{eq:3.4.11}). Поскольку $v''<0$ п.в., из леммы \ref{lem:3.3} следует, что строго убывающая функция $t\mapsto X_t$ абсолютно непрерывна.

Докажем теперь что стратегия (\ref{eq:3.4.4}), (\ref{eq:3.4.5}) является оптимальной для овыпукленной задачи (\ref{eq:3.2.17}).
Так как $v'$ абсолютно непрерывна (см. теорему \ref{th:3.1}), то функция $\ln v'(y)$ также абсолютно непрерывна на отрезке $[0,x]$. Следовательно,
$$ \ln v'(x)-\ln v'(y)=\int_y^x\frac{d}{dz}\ln v'(z)\,dz=\int_y^x\frac{v''(z)}{v'(z)}\,dz=\beta\int_y^x\frac{dz}{H'(v'(z))}, \quad y\in [0,x]$$
и
\begin{equation} \label{eq:3.4.12}
\frac{1}{\beta}\ln\frac{v'(x)}{v'(X_t)}=-t=-\int_{X_t}^x\frac{dz}{H'(v'(z))},\quad t\in [0,\tau].
\end{equation}
Из (\ref{eq:3.4.12}) по правилу дифференцирования сложной функции \cite[следствие 3.48]{Leo09} следует, что
$$\dot X_t=H'(v'(X_t))=\widehat\alpha(v'(X_t))-\widehat q(v'(X_t))=\alpha^*_t-q^*_t\quad\textrm{п.в. на } [0,\tau].$$
Здесь мы используем (\ref{eq:3.4.10}) и (\ref{eq:3.4.4}).
Для $t>\tau$ соотношение $\dot X_t=\alpha^*_t-q^*_t=0$ выполняется очевидным образом. Мы заключаем, что стратегия $(\alpha^*,q^*)$ допустима, так как она порождает неотрицательный фазовый процесс $X=X^{x,\alpha^*,q^*}$.

Чтобы доказать оптимальность стратегии $(\alpha^*,q^*)$ достаточно установить, что функция
\begin{equation} \label{eq:3.4.13}
W(t)=\int_0^t e^{-\beta s}(\widetilde R(q_s^*)-\widetilde C(\alpha_s^*))\,ds+e^{-\beta t} v(X_t^{x,\alpha^*,q^*})
\end{equation}
является константой, так как в этом случае
$$ W(0)=v(x)=\lim_{t\to\infty}W(t)=\int_0^\infty e^{-\beta s}(\widetilde R(q_s^*)-\widetilde C(\alpha_s^*))\,ds.$$
Дифференцируя (\ref{eq:3.4.13}), в силу (\ref{eq:3.4.4}) получаем
\begin{align*}
\dot W &=e^{-\beta t}\left(\widetilde R(q_t^*)-\widetilde C(\alpha_t^*)-\beta v(X_t^{x,\alpha^*,q^*})+v'(X_t^{x,\alpha^*, q^*})(\alpha_t^*- q_t^*)\right)\\
 &=e^{-\beta t}\left(-\beta v(X_t^{x,\alpha^*,q^*})+\widehat R(v'(X_t^{x,\alpha^*, q^*}))+\widehat C(v'(X_t^{x,\alpha^*, q^*}))\right)\\
 &=e^{-\beta t}\left(-\beta v(X_t^{x,\alpha^*,q^*})+H(v'(X_t^{x,\alpha^*,q^*}))\right)=0\quad \textrm{п.в. на } (0,\tau).
\end{align*}
Для $t>\tau$ имеем $X_t^{x,\alpha^*,q^*}=0$, $\alpha^*_t=q^*_t=\widehat u$ и
$$ W(t)=\int_0^\tau e^{-\beta s}(\widetilde R(q_s^*)-\widetilde C(\alpha_s^*))\,ds+\frac{1}{\beta}e^{-\beta\tau}(\widetilde R(\widehat u)-\widetilde C(\widehat u)),$$
так как $-e^{-\beta t}(\widetilde R(\widehat u)-\widetilde C(\widehat u))/\beta+e^{-\beta t}v(0)=0$ в силу оптимальности  $\widehat u$ для нулевого начального запаса: см. теорему \ref{th:3.4}.

Так как $v'(X_t)\in F$, $t\in\mathscr T$, то по лемме \ref{lem:3.6} заключаем, что соотношения (\ref{eq:3.4.6}) справедливы. Отсюда вытекает, что
$$ v(x)=\int_0^\tau e^{-\beta s}(R(q_s^*)-C(\alpha_s^*))\,ds+e^{-\beta \tau} v(0)$$
так как $W$ --- константа. Из теорем \ref{th:3.5} и \ref{th:3.4} следует, что стратегии (\ref{eq:3.4.4}), (\ref{eq:3.4.7}) и (\ref{eq:3.4.4}), (\ref{eq:3.4.8}) (при условии $\mathscr M_\zeta\neq \emptyset$) дают то же самое значение целевого функционала, что и (\ref{eq:3.4.4}), (\ref{eq:3.4.5}). Следовательно, они являются оптимальными для задач (\ref{eq:3.2.18}) и (\ref{eq:3.2.4}) соответственно.
\end{proof}

Заметим, что в силу (\ref{eq:3.4.10}) фазовый процесс $X^{x,\alpha^*,q^*}$, порожденный (\ref{eq:3.4.4}), является строго убывающим на $(0,\tau)$.
В случае $\mathscr M_\zeta=\emptyset$ можно использовать также приближенно оптимальную стратегию (\ref{eq:3.3.15}), (\ref{eq:3.3.16}) на $(\tau,\infty)$ вместо оптимальной распределенной стратегии (\ref{eq:3.4.7})

\section{Случай строго вогнутого мгновенного дохода и линейной функции затрат}\label{sec:3.5}
%\setcounter{equation}{0}
Пусть $Q=[0,\overline q]$, $A=[0,\overline\alpha]$, $\overline q$, $\overline\alpha>0$. Предположим что $R$ дифференцируема и строго вогнута, и пусть $C=c\alpha$, $c>0$. Оптимальная стратегия полностью описывается функциями $\widehat q(v'(x))$, $\widehat\alpha(v'(x))$ и значением $\widehat u$, указанным в теореме \ref{th:3.6}.

Имеем
$$ H(z)=\sup_{q\in [0,\overline q]}(R(q)-qz)+\sup_{\alpha\in [0,\overline\alpha]}(\alpha z-C(\alpha))
=\widehat R(z)+\overline\alpha(z-c)^+,\quad x^+=\max\{0,x\}.$$
Так как $\mathscr M_R(z)$ содержит в точности один элемент $\widehat q(z)$, то функция $\widehat R$ является непрерывно дифференцируемой (см. \cite[теорема 25.5]{Roc70}), и $\widehat q(z)=-\widehat R'(z)$ --- непрерывна.

Если $R'(0)\le 0$, то $\widehat q(z)=0$, $z\ge 0$, и $\zeta=\min\mathscr M_H=\{0\}$. Из теоремы \ref{th:3.1} следует, что $v(x)=H(0)/\beta=0$, и задача тривиальна. Таким образом, мы можем предполагать, что $R'(0)>0$.


Вычисление оптимальной статической стратегии (\ref{eq:3.3.1}) для нулевого оптимального запаса несложно:
$$ \{\widehat u\}=\arg\max_{u\in [0,\overline\alpha\wedge\overline q]}(R(u)-cu)=\begin{cases}
0,& R'(0)\le c,\\
(R')^{-1}(c),& R'(0)>c, R'(\overline\alpha\wedge\overline q)<c,\\
\overline\alpha\wedge\overline q,& R'(\overline\alpha\wedge\overline q)\ge c.
\end{cases}$$
Далее, из формулы
\begin{equation} \label{eq:3.5.1}
 H'(z)=\begin{cases}
-\widehat q(z),& z\in (0,c),\\
\overline\alpha-\widehat q(z),& z>c
\end{cases}
\end{equation}
следует, что $\widehat q$ --- неубывающая функция, так как $H$ выпукла. Ясно, что $\lim_{z\to+\infty}\widehat q(z)=0$ (и $\widehat q(z)=0$ для достаточно больших $z$, если $R'(0)<+\infty$). Из (\ref{eq:3.5.1}) заключаем, что
$$ \zeta=\min\mathscr M_H=\inf\{z\ge 0:\widehat q(z)=0\}\wedge\inf\{z\ge c:\widehat q(z)\le\overline\alpha\}>0.$$
Функция $\widehat q(v'(x))$, $x\in (0,\infty)$ является неубывающей и
$$ \lim_{x\to 0}\widehat q(v'(x))=\widehat q(v'(0))=\widehat q(\zeta)=\widehat u,$$
$$ \lim_{x\to +\infty}\widehat q(v'(x))=\widehat q(0)=\arg\max_{q\in [0,\overline q]} R(q)$$
в силу теорем \ref{th:3.1} и \ref{th:3.3}.

Функция $\widehat \alpha(v'(x))$, $x\in (0,\infty)$ является кусочно-постоянной:
$$ \widehat\alpha(v'(x))=\begin{cases}
\overline\alpha,& v'(x)>c,\\
0,& v'(x)<c.
\end{cases}$$
В частности, $\widehat\alpha(v'(x))=0$, $x>0$, если $\zeta=v'(0)<c$. Заметим, что в противоположном случае, когда стоимость производства $c$ ниже чем предельная непрямая полезность нулевого запаса: $\zeta=v'(0)>c$, производство должно запуститься, когда запас товара уменьшится до уровня $\widehat x>0$, определяемого уравнением $v'(\widehat x)=c$. Однако, после запуска производства уровень товарного запаса продолжает уменьшаться и достигает нуля за конечное время $\tau$.

Из (\ref{eq:3.4.9}) следует, что $v$ дважды непрерывно дифференцируема на $(0,\widehat x)\cup(\widehat x,+\infty)$, и $v''=\beta v'/H'(v')$ на этом множестве. Однако, $v''$ не является непрерывной в точке $\widehat x$:
$$ v''(\widehat x+0)=\frac{\beta c}{H'(c+)}\neq v''(\widehat x-0)=\frac{\beta c}{H'(c-)}.
$$
Таким образом, в предположениях теоремы \ref{th:3.1} функция $v''$ может быть разрывной.

\section{Пример Арвана-Мозеса}\label{sec:3.6}
%\setcounter{equation}{0}
В примере \cite{ArvMos81} спрос является линейным, следовательно получаемый мгновенный доход выглядит следующим образом:
$$ R(q)=(A-Bq)q,\quad q\in [0,A/B].$$
Функция затрат
$$ C(\alpha)=\alpha^3/3-K\alpha^2+K^2\alpha,\quad \alpha\ge 0$$
выпукла на $[0,K]$ и вогнута на $[K,\infty)$. Здесь предполагается что $A, B, K>0$.
Имеем,
$$ R'(q)=A-2Bq,\quad R''(q)=-2B,$$
$$ C'(\alpha)=(\alpha-K)^2,\quad C''(\alpha)=2(\alpha-K).$$
Заметим, что функция $C$ строго возрастающая и $1$-коэрцитивна. Функция $R$ строго вогнута.

Возьмем наибольшее $s$, для которого $s\alpha\le C(\alpha)$, $\alpha\ge 0$. Тогда
$$ s=\inf_{\alpha\ge 0}\{\alpha^2/3-K\alpha+K^2\}=\frac{K^2}{4},$$
где инфимум достигается при $\alpha_0=3K/2$. Легко видеть, что
$$ \widetilde C(\alpha)=(\co C)(\alpha)=\begin{cases}
K^2\alpha/4,&\alpha\in [0,3K/2],\\
C(\alpha),&\alpha\ge 3K/2.
\end{cases}$$

Для овыпукленной задачи (\ref{eq:3.2.17}) оптимальная статическая стратегия для нулевого начального запаса имеет вид
$$ \widehat u\in\arg\min_{u\in [0,A/B]}(R(u)-\widetilde C(u)).$$
Из равенств
$$ R'(u)-\widetilde C'(u)=\begin{cases}
A-2Bu-K^2/4,& u<3K/2,\\
A-2Bu-(u-K)^2,& u>3K/2
\end{cases}$$
следует, что
\begin{equation} \label{eq:3.6.1}
\widehat u=\begin{cases}
0,& A\le K^2/4,\\
(A-K^2/4)/(2B),& K^2/4\le A\le 3BK+K^2/4,\\
-B+K+\sqrt{B^2-2BK+A},& A\ge 3BK+K^2/4.
\end{cases}
\end{equation}

Для $A\le K^2/4$ и $A\ge 3BK+K^2/4$ имеем $\widetilde C(\widehat u)=C(\widehat u)$. Следовательно, в этих случаях $\alpha_t=q_t=\widehat u$ является оптимальным решением исходной задачи (\ref{eq:3.3.4}). Если
\begin{equation} \label{eq:3.6.2}
\frac{K^2}{4}< A< 3BK+\frac{K^2}{4},
\end{equation}
то $\widehat u$ принадлежит интервалу $(0,3K/2)$, где $\widetilde C$ линейна и $\widetilde C<C$. В этом случае оптимальная распределенная стратегия производства определяется формулами (\ref{eq:3.3.9}) -- (\ref{eq:3.3.11}):
\begin{equation} \label{eq:3.6.3}
\overline\alpha_t(dx)=\nu\delta_0+(1-\nu)\delta_{3K/2},\quad (1-\nu)\frac{3K}{2}=\widehat u=\left(A-\frac{K^2}{4}\right)\frac{1}{2B}.
\end{equation}

Таким образом, для нулевого начального запаса обычная статическая стратегия не является оптимальной, если и только если выполнено условие (\ref{eq:3.6.2}). В этом случае вместо распределенной стратегии (\ref{eq:3.6.3}) можно использовать приближенно оптимальную стратегию (\ref{eq:3.3.15}), (\ref{eq:3.3.16}). При использовании этой стратегии запас остается близким к $0$: $X_t\le b\varepsilon$, $b>0$, и демонстрирует циклическое поведение накопления-сокращения, описанное в \cite{ArvMos81}. Однако, оно может не производить дисконтированную прибыль близкую к оптимальной, если циклы накопления-сокращения не малы. Также интересно отметить, что условие
$$ K^2< A< 3BK+\frac{K^2}{4},$$
весьма похожее на (\ref{eq:3.6.2}), было указано в работе \cite{ArvMos81}.

Пусть $z\ge 0$. Обозначим через $\widehat\alpha$, $\widehat q$ точки максимума функций
$$ z\alpha-\widetilde C(\alpha)\to\max_{\alpha\ge 0},\quad R(q)-z q\to\max_{q\in [0,A/B]}.$$
Имеем,
\begin{equation} \label{eq:3.6.4}
\widehat\alpha(z)=\begin{cases}
    0,      &\quad z\in [0,K^2/4),\\
    K+\sqrt z, & \quad z> K^2/4,
  \end{cases}\quad
\widehat q(z)=\begin{cases}
    (A-z)/(2B),      &\quad z\in [0,A],\\
    0,       & \quad z\ge A,
  \end{cases}
\end{equation}
и $\widehat\alpha(K^2/4)\in [0,3K/2]$. Найдем наименьшую точку минимума $\zeta=\min\mathscr M_H$ гамильтониана
\begin{align*}
H(z)&=\sup_{q\in [0, A/B]}\{(A-B q) q-zq\}+\sup_{\alpha\ge 0}\{z\alpha-\widetilde C(\alpha)\}\\
 &=\frac{(z-A)^2}{4B} I_{\{z< A\}}+(z\widehat\alpha(z)-C(\widehat\alpha(z))) I_{\{z> K^2/4\}},\quad z\ge 0.
\end{align*}
Если $A\le K^2/4$, то $\zeta=A$. В противном случае, рассмотрим
$$ H'(z)=\frac{z-A}{2B}+\widehat\alpha(z)=\frac{z-A}{2B}+K+\sqrt z,\quad z\in \left(\frac{K^2}{4},A\right).$$
Поскольку $\lim_{z\nearrow A} H'(z)>0$, то отсюда следует, что $\zeta\in (K^2/4,A)$, если и только если
$$ \lim_{z\searrow K^2/4}=\frac{1}{2B}\left(\frac{K^2}{4}-A+3BK\right)<0$$
При этом условии $\zeta$ определяется уравнением $H'(\zeta)=0$, $\zeta\in (K^2/4,A)$. В противном случае, $\zeta=K^2/4$. Объединяя все случаи, рассмотренные выше, получаем
$$\zeta=\begin{cases}
A,& A\le K^2/4,\\
K^2/4,& K^2/4\le A \le K^2/4+3BK,\\
\left(-B+\sqrt{B^2-2BK+A}\right)^2,& A\ge K^2/4+3BK.
\end{cases}
$$
Заметим, что эти три случая совпадают со случаями, указанными в (\ref{eq:3.6.1}).

Рассмотрим стратегии с обратной связью $\widehat q(v'(x))$, $\widehat\alpha(v'(x))$, определенные в теореме \ref{th:3.6}. Так как $v'(x)\le v'(0)=\zeta$ и $\zeta\le A$ для любого набора параметров, то справедливо следующее равенство
$$ \widehat q(v'(x))=\frac{A-v'(x)}{2B},\quad x>0.$$
Следовательно, $\widehat q(v'(x))$, $x>0$ является строго возрастающей положительной функцией. Кроме того, легко видеть, что
$$ \lim_{x\searrow 0}\widehat q(v'(x))=\frac{A-\zeta}{2B}=\widehat u.$$

Если $A\le K^2/4+3BK$, то $\zeta\le K^2/4$ и
$$ \widehat\alpha(v'(x))=0,\quad x>0.$$
Если $A>K^2/4+3BK$, то $v'(0)=\zeta>K^2/4$, и существует единственная точка $\widehat x>0$ такая, что $v'(\widehat x)=K^2/4$. В этом случае
$$ \widehat\alpha(v'(x))=\begin{cases}
K+\sqrt{v'(x)},& x<\widehat x,\\
0,& x>\widehat x.
\end{cases}$$

Таким образом, имеются три случая. 

(i) Если $A\le K^2/4$, то фирма должна оптимально продать начальный запас:
$$ \dot X_t=-\widehat q(v'(X_t))=-\frac{A-v'(X_t)}{2B}<0,\quad X_t>0.$$
Производство не отсутствует.

(ii) Если $\quad K^2/4<A< K^2/4+3BK$, то производство начинается после продажи начального запаса, и распределенная производственная стратегия (\ref{eq:3.6.3}) должна соответствовать спросу $\widehat q(\zeta)=\widehat u=(A-K^2/4)/2B$.

(iii) Если $A\ge K^2/4+3BK$, то производство начинается после того, как уровень товарного запаса падает ниже $\widehat x$. При этом товарный запас продолжает уменьшаться до $0$
и стабилизируется на этом уровне. Оптимальные спрос и интенсивность производства соответствующего устойчивого режима равны значению $\widehat u$, определенному в (\ref{eq:3.6.1}).


Наконец, чтобы проиллюстрировать результат теоремы \ref{th:3.3} покажем, что $\mathscr M_\eta\neq\emptyset$, если и только если выполняется условие (\ref{eq:3.6.2}). Пусть $\varphi(\alpha,z)=z\alpha-C(\alpha)$. Рассматривая
$$ \frac{\partial\varphi}{\partial\alpha}(\alpha,z)=z-(\alpha-K)^2,\quad \alpha>0,$$
заключаем, что $\varphi(\cdot,z)$ имеет две локальных точки максимума $\alpha_1=0$, $\alpha_2=K+\sqrt z$ при $z\in (0, K^2)$ и точку глобального максимума $\alpha_2=K+\sqrt z$ при $z\ge K^2$. Далее,
$$\frac{d}{dz}\varphi(\alpha_2(z),z)=\alpha_2(z)+(z-(\alpha_2(z)-K)^2)\alpha_2'(z) =\alpha_2(z)>0,\quad z>0$$
и $\varphi(\alpha_2(K^2/4),K^2/4)=\varphi(3K/2,K^2/4)=0.$
Отсюда следует что
$$\varphi(\alpha_1(z),\eta)=0>\varphi(\alpha_2(z),z)\quad \iff\quad z<K^2/4$$
и
$$ \mathscr M_C(z)=\arg\max_{\alpha\ge 0}(z\alpha-C(\alpha))=
\begin{cases}
0,& z\in [0,K^2/4),\\
\{0,3K/2\},& z=K^2/4,\\
K+\sqrt z,& z>K^2/4.
\end{cases}$$

Ясно, что $\mathscr M_\eta\neq\emptyset$, $\eta\ge 0$, если и только если
$\widehat q(\eta)\not\in\mathscr M_C(\eta)$, $\eta\ge 0,$ где $\widehat q(\eta)$ определяется уравнением (\ref{eq:3.6.4}).
Легко видеть, что этот случай имеет место, если и только если $A>K^2/4$ и
$$\widehat q(K^2/4)=\frac{A-K^2/4}{2B}<\frac{3K}{2},$$
что равносильно правому неравенству (\ref{eq:3.6.2}). Таким образом, из теоремы \ref{th:3.3} опять следует, что не существует оптимальной обычной стационарной стратегии для нулевого начального товарного запаса, если и только если выполнены неравенства (\ref{eq:3.6.2}).
\clearpage 